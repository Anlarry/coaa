\documentclass[UTF8]{ctexart}


% \lstset{frame=, basicstyle={\footnotesize\ttfamily}}



% \graphicspath{ {images/} }
\usepackage{ctex}
\usepackage{minted}
\usepackage[colorlinks,
            linkcolor=black,       %%修改此处为你想要的颜色
            anchorcolor=black,  %%修改此处为你想要的颜色
            citecolor=black,        %%修改此处为你想要的颜色,例如修改blue为red
            ]{hyperref}
%-----------------------------------------BEGIN DOC----------------------------------------

\begin{document}
\renewcommand{\contentsname}{目\ 录}
\renewcommand{\appendixname}{附录}
% \renewcommand{\appendixpagename}{附录}
\renewcommand{\refname}{参考文献} 
\renewcommand{\figurename}{图}
\renewcommand{\tablename}{表}
\renewcommand{\today}{\number\year 年 \number\month 月 \number\day 日}

\title{{\Huge U10M11007试点班实验报告{\large\linebreak\\}}{\Large 指令分析报告/XX设计报告\linebreak\linebreak}}
%please write your name, Student #, and Class # in Authors, student ID, and class # respectively
\author{\\姓\ 名:王\ 嘉\ 利\\
学\ 号: 2018302278\\
班\ 号: 10011801\\\\
CS 11007 计算机组成与体系结构\\
(春季, 2020)\\\\
西北工业大学\\
计算机学院\\
ERCESI}
\date{\today}
\maketitle
\newpage

%-----------------------------------------ABSTRACT-------------------------------------
\begin{center}
{\Large\bf{摘\ 要\\}}
\end{center}
这次实验主要整理和分析了SMIPS指令,将MIPS指令整理在EXCEL中,并且简单分析了数据通路特征
\newpage
%-----------------------------------------ABSTRACT-------------------------------------
\begin{center}
{\Large\bf{版\ 权\ 声\ 明\\}}
\end{center}
该文件受《中华人名共和国著作权法》的保护。ERCESI实验室保留拒绝授权违法复制该文件的权利。任何收存和保管本文件各种版本的单位和个人,未经ERCESI实验室(西北工业大学)同意,不得将本文档转借他人,亦不得随意复制、抄录、拍照或以任何方式传播。 否则,引起有碍著作权之问题,将可能承担法律责任。\newpage
%-----------------------------------------CONTENT-------------------------------------
\begin{center}
\tableofcontents\label{c}
\end{center}
\newpage

%------------------------------------------TEXT--------------------------------------------

%----------------------------------------OVERVIEW-----------------------------------------

\section{概述} \label{overview}%------------------------------
SMIPS指令可以大概分为运算指令(逻辑运算、算术运算、移位运算)、分支跳转指令、访存指令、数据移动指令、和自陷指令。
因为指令的数量是比较少的,所以可以从指令
的编码规律来考虑译码。再按不同指令类别考虑数据通路。
%----------------------------------SYSTEM DESIGN------------------------------------------

\newpage
\section{指令译码} \label{sysdes}%------------------------------
对于R-type指令,区分指令依赖于func段的编码,而对于I-type和J-type依赖opcode段。
不难发现,可以将func段和opcode段分成前三位和后三位来译码。
\subsection{运算指令}\label{sub:sysover}
算术运算指令由R-type和I-type两类指令组成
\subsubsection{算术运算 \ 逻辑运算}\label{sub:Interface}

\textbf{R-type}

可以将R-type的算术运算和逻辑运算分为

\begin{minted}{c}
    class1 func[5:3] 100 : ADD ADDU SUB SUBU   AND NOR OR XOR
    class2 func[5:3] 101 : SLT SLTU
    class3 func[5:3] 011 : DIV DIVU MULT MULTU 
\end{minted}

对于第一类,\mintinline{c}{func[2]}可以将其分为算术运算和逻辑运算,对于算术运算
\mintinline{c}{func[1]}区分加减法,\mintinline{c}{func[0]}区分有无符号。
\\

对于其他两类都是算术运算,\mintinline{c}{SLT SLTU}将比较结果写回\mintinline{c}{rt}寄存器
,其\mintinline{c}{func[2:0]}与
相应的\mintinline{c}{SUB SUBU}相同;对于乘法和除法指令
再数据通路上与class1,class2有差异,是将计算的结果写回到\mintinline{c}{HI LO}寄存器
\\

另外,区分\mintinline{c}{ADD(U) SUB(U) SLT(U) DIV(U) MULT(U)}的有无符号都可以用
\mintinline{c}{func[0]} \\

\textbf{I-type}

可以发现下面这些指令的\mintinline{c}{opcode[5:3]}都是\mintinline{c}{001}

\begin{minted}{c}
    ADDI ADDIU SLTI SLTIU 
    ANDI LUI   ORI  XORI
\end{minted}

另外可以发现除了\mintinline{c}{LUI},其他指令的\mintinline{c}{opcode[2:0]}可R-type中相同
功能的指令的\mintinline{c}{func[2:0]}是一样的\\
\par\textbf{LUI}指令 \ 与其他指令一些差异,写回\mintinline{c}{rt}寄存器的值是$imm||0_{16}$,因此
在数据通路可能需要一个单元完成数值的拼接。

\subsubsection{移位运算}

移位运算指令是\mintinline{c}{SLLV SLL SRAV SRA SRLV SRL},可以发现他们的\mintinline{c}{opcode[5:3]}
都是0,\mintinline{c}{opcode[2]}区分使用\mintinline{c}{sa}还是\mintinline{c}{R[rs]}作为移位量
,\mintinline{c}{opcode[1]}区分左移和右移,\mintinline{c}{opcode[0]}区分逻辑移位还是算术移位


\subsection{分支指令}

\subsubsection{无条件跳转}

有四个无条件跳转指令\mintinline{c}{JR JALR J JAL}

对于\mintinline{c}{JR JALR}是R-type指令,\mintinline{c}{func[5:3]}都是\mintinline{c}{001};
对于\mintinline{c}{J JAL}是J-type指令,\mintinline{c}{opcode[5:3]}都是0

另外\mintinline{c}{func[0]}(或者\mintinline{c}{opcode[0]})区分了是否要写\mintinline{c}{rt}寄存器


\subsubsection{有条件跳转}

\begin{minted}{c}
    BEQ  BNE  BGEZ   BQTZ 
    BLEZ BLTZ BGEZAL BLTZAL
\end{minted}

可以发现这些指令的\mintinline{c}{opcode[5:3]}都是0(与\mintinline{c}{J JAL}在这一点上相似)
\\

需要注意的是只有\mintinline{c}{BEQ BNE}是需要读\mintinline{c}{rs rt}寄存器的,其他指令只需要读
\mintinline{c}{rs},而且\mintinline{c}{rt}段时给定的。另外对于\mintinline{c}{BGEZ BLTZ BGEZAL BLTZAL}
这四个指令的\mintinline{c}{op}段都是\mintinline{c}{000 001},要靠\mintinline{c}{rt}段来区分

\subsection{访存指令}

\begin{minted}{c}
    LB LBU LH LHU LW
    SB SH SW
\end{minted}

对于lowd指令\mintinline{c}{opcode[5:3]}都是\mintinline{c}{100},\mintinline{c}{opcode[2]}区分了有符号扩展和无符号扩展

对于store指令\mintinline{c}{opcode[5:3]}则是\mintinline{c}{101}

\subsection{数据移动指令}

\begin{minted}{c}
    MFHI MFLO MTHI MTLO
\end{minted}

这四个指令是R-type指令,\mintinline{c}{func[5:3]=010},\mintinline{c}{func[1]}区分Hi、LO,\mintinline{c}{func[0]}区分
数据移动的方向

\subsection{特权指令和自陷指令}

\begin{minted}{python}      
    特权指令  ERET MFC0 MTC0  opcode=010 000
    自陷指令  BREAK SYSCALL   func[5:3]=001
\end{minted}


% -----------------------------------BLOCKS DESIGN----------------------------------------
\section{数据通路特征}

\subsection{运算指令}

\textbf{R-type}

对于R-type类型的运算指令的数据通路的共同特征可以总结为,从RF中取出\mintinline{c}{rt rs}的值,
\mintinline{c}{func}段指定ALU的运算规则,将结果写回rd(乘法和除法除外,乘除的结果写回HI LO寄存器,而且其他指令的
计算结果是32bit)

\textbf{I-type}

对于I-type类型的运算指令有两类,一类是算术运算,另一类是逻辑运算。对于算术运算\mintinline{c}{Imm}扩展都是符号扩展
,对于逻辑运算\mintinline{c}{Imm}扩展都是无符号扩展。在指令译码部分已经提到了这两类的\mintinline{c}{opcode[5:3]=001},
而\mintinline{c}{opcode[1:0]}进一步功能划分,可以发现\mintinline{c}{opcode[2]}是可以作为符号扩展
还是无符号扩展的标志

I-type类型的数据通路特征则是,从RF中取出\mintinline{c}{rs},将立即数扩展至32bit,将这两个数送入ALU得到一个输出,
将运算结果写回rt(对于LUI比较特殊,可能可以认为是从\$0取出0,和Imm运算,因为对于其他情况都要扩展立即数,所以,可能还可以认为
LUI也是将Imm进行扩展后送入ALU,这样就可以简化设计)


\subsection{分支跳转}

\subsubsection{有条件跳转}

这类指令需要从RF中读出rs的值,对于\mintinline{c}{BEQ BNE}还需要读出rt的值,对于其他六个指令不需要。
另外,在译码阶段还需要把offset进行符号扩展。在执行阶段算出是是否要跳转,产生一个选择信号。
对于\mintinline{c}{BGEZAL BLTZAL}还需要在写回阶段将\mintinline{c}{pc+8}写回\$31

\subsubsection{无条件跳转}

对于\mintinline{c}{J JAL}不需要在译码阶段读寄存器,但是需要将\mintinline{c}{target}和pc拼接成32bit;
对于\mintinline{c}{JR JALR}在译码阶段读寄存器rs得到跳转目标。对于\mintinline{c}{JAL JALR}需要在写回阶段将\mintinline{c}{pc+8}写回\$31

\subsection{访存指令}

\subsubsection{load}

在译码阶段需要从读出base寄存的值,并对offset进行有符号扩展,在执行阶段得到地址,在访存阶段得到要写到rt寄存器的数据,
在写回阶段将读出的数据写回寄存器。因为读出的数据可以一个字节、半字、字,也就是有\mintinline{c}{LB LH LW}等这三种指令,对读出的数据要进行符号扩展,
那么在数据通路中就应该再有一个扩展单元,把从内存读到的数扩展。

\subsubsection{store}

在译码阶段、执行阶段和load类似,在访存阶段需要把数据rt寄存器的一个字节、半字或字写入到存储器。

\subsection{数据移动指令}

\mintinline{c}{MFHI MFLO}需要在写回阶段把HI或LO寄存器中的值写回到rd寄存器

\mintinline{c}{MTHI MTLO}在译码阶段读出rs的值,在写回阶段写回到rd寄存器

\subsection{关于PC}

每拍需要从跳转指令的跳转目标和PC+4中选择一个(要考虑ERET)

\subsection{关于RF}

通过上述的分析可以确定RF需要一个写端口和两个读端口,写入的数据可能来自译码阶段的立即数扩展(LUI)、ALU计算的结果、
存储器中读出的值、来自HI,LO寄存器或者来自协处理器。

\subsection{关于存储器}

要提供一个地址接口,输出接口,写入数据端口(还有WRITE\_EN)。地址来自于对offset扩展。
另外还需要考虑\mintinline{c}{LB LH LW},以及\mintinline{v}{SB SH SW},他们在读出、和写入的位宽不一样。



%add more subsections for other block in you CPU design.

\section{实验过程记录}\label{labrec}

将SMIPS的指令按R-type,I-type和J-type整理,以及按功能整理

\subsection{问题与解决方案}

\textbf{问题}\ \ 对于MIPS怎么处理异常、协处理器怎么控制CPU,不是特别了解

\textbf{解决方案} \ \ 查看MIPS手册了解了大概
 % -----------------------------------Appendix----------------------------------------
\appendix
% \section{代码}\label{sub:app.code}
% 请在附录\ref{sub:app.code}中添加代码。请使用如下Scala的语法高亮描述方法。

\newpage
% -----------------------------------REFERENCE----------------------------------------
\begin{thebibliography}{9}
	MIPS Architecture For Programmers Volume I  \\
	MIPS Architecture For Programmers Volume II \\
	MIPS Architecture For Programmers Volume III \\
	SEE MIPS RUN \\
\end{thebibliography}
\end{document}

